%%%%%%%%%%%%%%%%%%%%%%%%%%%%%%%%%%%%%%%%
%%%%%%%%%%%%%%%%POLICY%%%%%%%%%%%%%%%%%%
%%%%%%%%%%%%%%%%%%%%%%%%%%%%%%%%%%%%%%%%
\section{Policy Background}
    \subsection{Abortion Access \& The Hyde Amendment}
        The first major change in US abortion access came with the ruling of \textit{Roe v. Wade}. Prior to this ruling, abortion was legal in only five states (Alaska, California, Hawaii, New York, Washington). Then from 1973 to 1976, all individuals living inside the U.S. had federally unrestricted access to abortions.

        On September 30$^{th}$, 1976, the Ford administration passed the first federal law in an attempt to constrain the outcome of \textit{Roe v. Wade}. The Hyde Amendment was originally passed as a law preventing federal funding to be used to directly fund an abortion by preventing the use of Medicaid funding to help finance an abortion except in cases where the pregnancy endangers the woman's life.
        Since 1976, the Hyde Amendment has been altered twice by federal policy but remains intact as an abortion-restricting policy today. In 1993, the Clinton administration passed the \textit{Departments of Labor, Health, Human Services, and Education, and Related Agencies Appropriation Act, 1994} which added the cases of pregnancies due to rape or incest to the list of exceptions (\cite{clinton}). In 2010, the Obama administration issued Executive Order 13535, extending the Hyde Amendment to remain intact under the Affordable Care Act (ACA) (\cite{obama}). Currently, the Trump administration is attempting to pass H.R. 7 (No Taxpayer Funding for Abortion and Abortion Insurance Full Disclosure Act of 2017) in order to permanently block taxpayer money from funding abortions (\cite{trump}).

        Forty-one years later, the Hyde Amendment still restricts abortion access for low-income individuals and, by extension, the decision to have a child. Medicaid continues to be an essential source of health coverage for over 74 million people (\cite{1_in_5}). One in five women in reproductive age (15 to 44) are covered by Medicaid, totaling nearly 13 million women who are negatively impacted by the Hyde Amendment (\cite{1_in_5}). This leaves the potential for many unwanted pregnancies with no option but to carry to term, regardless of the financial stability, family situation or readiness of the parents.
