%%%%%%%%%%%%%%%%%%%%%%%%%%%%%%%%%%%%%%%%
%%%%%%%%%%%%%%%%INTRO%%%%%%%%%%%%%%%%%%%
%%%%%%%%%%%%%%%%%%%%%%%%%%%%%%%%%%%%%%%%
\section{Introduction}
    For the past century, abortion access ranked among the most controversial social issues facing America. Even today, the country is strongly divided between the `Right to Live' and the `Right to Choose.' Whether it be Planned Parenthood scandals, Trump's promises to nominate a supreme court justice that will overturn Roe v. Wade or the Trump administration's attempt to cut federal funding to Planned Parenthood, it is clear that this issue of abortion access is far from resolved.

    Abortion access can come in many shapes and sizes, creating a dynamic arena for policy and law. We have private providers that offer abortion services to those who can pay for it and public hospitals with entire divisions dedicated to these cases. Planned Parenthood plays a huge role in providing abortion services for women, ranging from funding to awareness and education and even clinics ready to perform abortions when needed.

    Changes to abortion access also come in many different varieties. Beginning with \textit{Roe v. Wade}, state and federal laws have since created a constantly changing and increasingly complicated environment for individuals who want an abortion. Now, when considering an abortion, it's not enough to simply consider financial feasibility and location of the closest clinic. Now you have to consider restrictions such as physician and hospital requirements, gestational limits, public funding, insurance coverage, state-mandated counseling, waiting periods, parental involvement, and the list goes on. On top of that, not all states have similar combinations of abortion access policies. While some states require that a woman must go through counseling on topics such as ``the ability for a fetus to feel pain,'' (\cite{guttmacher_3}) others require that the woman waits 24 hours between the initial appointment and the procedure.

    While all of these policies and red tape seems complicated enough for individuals who seeks out an abortion, economists have taken up the task to observe changes in decision making and unintended consequences resulting from recent changes to abortion access. For example, prior to 1973 abortions were illegal in all but 5 states. Then on January $23^{rd}$, 1973, abortions suddenly became legal for every U.S. citizen. Intuitively it makes sense that decision-making process would change as a result of this ruling, and in the years following we see this in the data. Following the increases in abortion access seen in the early 1970’s, birth rates dropped by as much as 10\% (\cite{levine_2}).
    Furthermore, economists began finding externalities unrelated to the purpose of these policy changes. Specifically, the economic literature shows that the liberalization of abortion access tended to elicit positive outcomes for future generations. The average standard of living for children born after \textit{Roe v. Wade} begins to increase (\cite{gruber}). As a result, children are less likely to become involved in crime (\cite{levitt}), consume more education and rely less on welfare (\cite{ananat}).

    In this paper, I look to expand the literature on long-lasting impacts of abortion policy by studying the impact of the Hyde Amendment on future education outcomes. I use data collected from the 2010 American Community Survey (ACS) and Medicaid expenditure data collected from the Centers for Medicare \& Medicaid Services (CMS). I use state Medicaid expenditures per person and state Medicaid expenditures per enrollee as proxies to measure the importance of Medicaid allocations on the decision to have an abortion. I create a dummy variable to indicate whether or not an individual's birth is impacted by the Hyde Amendment. I regress an interactive variable of the Hyde Amendment and the state Medicaid expenditures to analyze the impact of the Hyde Amendment on educational attainment for the future generation.

    Results show that the coefficient for the proxy interaction variable is statistically significant and positive, suggesting that the Hyde Amendment resulted in positive selection. This result goes against the economic theory and literature surrounding abortion access and future outcomes. I conclude that this result is likely due to poor model specification and bias induced from omitted variables.
