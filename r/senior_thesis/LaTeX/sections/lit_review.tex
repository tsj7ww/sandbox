%%%%%%%%%%%%%%%%%%%%%%%%%%%%%%%%%%%%%%%%
%%%%%%%%%%%%%%LIT_REVIEW%%%%%%%%%%%%%%%%
%%%%%%%%%%%%%%%%%%%%%%%%%%%%%%%%%%%%%%%%
\section{Literature Review}
    \subsection{The Marginal Child \& Selection}
        In abortion policy analysis, the concept of the `marginal child' is a framework used to understand the changes in standards of living resulting from abortion policy changes. Several papers in the literature have discussed and analyzed the concept of the marginal child by addressing the question: how does abortion access change the decision to have a child? Economics can shine a light on this question by focusing on the selection problem.

        Consider the world before \textit{Roe v. Wade}. Any law-abiding woman who found herself pregnant must have the child. In this situation, there was no selection; there was no decision to be made because there was only one option once pregnant: have the child. After \textit{Roe v. Wade}, selection came into play by allowing for two options. Given that you were pregnant, you were then able to choose to have the child or not have the child. In this new realm of decision-making with selection, the marginal child is defined as the child who is not born because of increases in abortion access.

        Given the new realm of decision making with selection, what happens to the demographic of individuals born post-\textit{Roe v. Wade}? Applying the marginal child as a framework, we can calculate the living standard of the average child and compare it to the predicted standard of living of the marginal child of the same cohort after increases in abortion access. To understand this framework, it is best to consider two different scenarios.

        In scenario one, we have two individuals who find themselves pregnant and would like to have a child. Individual (1) is not financially ready to have a child while individual (2) is. In this scenario, we would expect individual (1) to have an abortion and individual (2) to carry to term, leaving the average child born in better circumstances than the marginal child. This type of selection is called \textit{positive selection}.

        In scenario two, we have individual (3) who is not financially ready to have a child and individual (4) who is. In addition, individual (4) did not expect this pregnancy and would not like to carry to term and individual (3) does not have the finances to fund an abortion. In this scenario, although both individuals would prefer an abortion, only individual (4) is able to have one, leaving the average child born in worse circumstances than the marginal child. This type of selection is called \textit{negative selection}.

        This framework of selection and the marginal child allows economists to observe how changes to abortion access impact outcomes for future generations. Gruber et al. (1998) take on this task by asking a general question: would the children who are not born because of abortion access have lived in better or worse circumstances than the average child in their cohort? Using differences-in-differences, Gruber takes advantage of the two natural experiments that happened in the 1970's. The first natural experiment being the 5 states who independently repealed anti-abortion laws and the second being the \textit{Roe v. Wade} ruling.
        The results show a comprehensive increase in the standard of living for the average child, suggesting that increases in abortion access promote positive selection. The marginal child would have been 35\% more likely to die as an infant, 40\% more likely to live in poverty, 50\% more likely to receive welfare, and 70\% more likely to live in a single parent home. Furthermore, their estimates show that positive selection and decreasing individuals reliant on government welfare saved the government over \$14 billion in welfare expenditures from 1970 to 1994 (\cite{gruber}).

    \subsection{Selection \& Future Outcomes}
        While Gruber and many other economists give insight to the impact of the marginal child in terms of living circumstances, another area to consider is the impact of abortion access on later-life outcomes. Extending the use of the natural experiments seen in the early 1970's, Donohue and Levitt (2001) famously tie together trends in abortion access and crime rates. With similar methodology used by Gruber, Donohue and Levitt show that the change in abortion access during the early 1970's led to substantial decreases in crime rates seen in the early 1990's with positive selection again being the mechanism for this result. From positive selection, the average child's standard of living increases in comparison to the marginal child. Due to the increased living standards, children born in the early to mid-1970's are raised in conditions that are less conducive to participating in crime later on in life, making the connection between abortion access and crime rates (\cite{levitt}).

        Not all of the literature supports positive selection resulting from increased abortion access. Davido (2009) again uses the natural experiment seen in the early 1970's to base the claim that abortion access is not necessarily the main contributing factor to changes in later-life outcomes. Using a slightly modified model specification created by Ananat et al., a study that concludes that abortion access is positively correlated with enhances in later-life outcomes (\cite{ananat}), Davido proposes two alternative reasons for this trend. First, Davido claims that that the law changes are not exogenous and that liberal states tended towards early legalization. Although abortion became legally acceptable after \textit{Roe v. Wade}, it likely only became socially acceptable in the liberal states. Second, because the early legalization states tended to be large and wealthy, that sample does not represent a random treatment of all states, possibly creating problems with the coefficients (\cite{davido}).
