%%%%%%%%%%%%%%%%%%%%%%%%%%%%%%%%%%%%%%%%
%%%%%%%%%%%%%%%CONCLUSION%%%%%%%%%%%%%%%
%%%%%%%%%%%%%%%%%%%%%%%%%%%%%%%%%%%%%%%%
\section{Conclusion}
    By using a proxy to measure dependence on Medicaid, the results of my analysis show that the Hyde Amendment had a significant impact on the education outcomes for the marginal child. The significant, positive coefficient on the proxy variable suggests that the Hyde Amendment promoted positive selection, contrary to what economic theory predicts would happen. 

    There are two aspects of the coefficient that possibly give light to why the analysis is in discordance with economic theory. First, even though the coefficients are significantly positive, they are also very close to zero. While the coefficients are statistically significant, the size of the coefficients suggests no practical significance. Second, because of the unavailability of more specific and descriptive data in the realm of professional health services, it is likely that bias induced by unobservable factors is skewing the results. For example, consider California and Mississippi. California has a much higher level of Medicaid expenditures per enrollee than Mississippi. One characteristic unobserved in this model specification is the level of state education spending, which is likely positively correlated with the level of Medicaid expenditures. Given that this is the case, the proxy variable would be measuring the dependence of a state's population on Medicaid with positive bias from the positive correlation between state Medicaid expenditures and state education expenditures. Going forward, this analysis could be improved by including more individual-oriented health data as well as an enhanced model specification.

    \newpage
